%----------------------------------------------------------------------
% Problem 2

\begingroup
\allowdisplaybreaks

\newpage
\section{Problem 2}

\textbf{Exercise 11 in Section 11.6}

\subsection{Solution}

\subsubsection{Part A}

To call the \verb|mcmc()| function, I adjusted the function \verb|logprior()| to \verb|logprior1| which returns $0$ if the candidate model is within our specified grid, and $-\inf$ otherwise to be consistent with a uniform distribution. I also adjusted the function \verb|loglikelihood| to return the likelihood for a given candidate model for our problem. 

I ran $10,000$ samples with a step size of half of the steps used to build the grid. I tried to run more samples, but it caused my laptop to crash when I let it run over night. For an initial model, I choose to just start in the middle of the grid to see what would happen. 

\begin{figure}[h] 
	\centering
	\includegraphics[width=0.75\textwidth]{./images/prob2_uniform_prior.eps}
	\caption{MCMC for a Uniform Prior}
	\label{fig: prob2 mcmc uniform prior}
\end{figure}
\FloatBarrier

The resulting distribution from the samples appears looks similar to the posterior distribution we computed in the previous problem! 


\subsubsection{Part B}

To approximate the marginal probabilities, I simply plotted a histogram of each row of the candidate model history from the \verb|mcmc()| function output. In \MATLAB, I called the \verb|histogram()| function and used the \verb|"pdf"| option for normalization. This resulted in the figure below. 

\begin{figure}[h] 
	\centering
	\includegraphics[width=0.75\textwidth]{./images/prob2_uniform_prior_marginal_prob.eps}
	\caption{Marginal Probabilities with a Normal Prior}
	\label{fig: prob2 mcmc uniform prior marginal prob}
\end{figure}
\FloatBarrier

These marginal probabilities don't match the previous problem, but that might be due to a low amount of samples.


\subsubsection{Part C}

I repeated the same process as parts A and B, however I made a new function titled \verb|logprior2| that computed the value of the probability density function for the candidate value of $\epsilon$. This resulted in the figure below. 

\begin{figure}[h] 
	\centering
	\includegraphics[width=0.75\textwidth]{./images/prob2_normal_prior.eps}
	\caption{MCMC for a Uniform Prior}
	\label{fig: prob2 mcmc normal prior}
\end{figure}
\FloatBarrier

Using the same process for the marginal probabilities, I approximated the marginal probabilities below. 

\begin{figure}[h] 
	\centering
	\includegraphics[width=0.75\textwidth]{./images/prob2_normal_prior_marginal_prob.eps}
	\caption{Marginal Probabilities with a Normal Prior}
	\label{fig: prob2 mcmc normal prior marginal prob}
\end{figure}
\FloatBarrier


\subsubsection{Part D}

It took me quite a bit of time to figure out how to apply MCMC to this problem, but I am proud that I was able to figure it out. It took a lot of tuning for the number of samples and the step size to begin to see how using this MCMC method provided similar results to what we saw in the first problem. 

With the uniform prior, the samples seemed to congregate outside of the original search grid that was established in the last homework assignment. The marginal probability for $\epsilon$ seems to have a similar shape as the last problem, but the histogram contains values outside of the search grid. 

With the normal prior, the original scatter plot seemed to have the same distribution as the previous problem which to me appeared as a source of validation for the MCMC method being implemented properly. However, inspecting the marginal probabilities, it appears that both distributions are now bimodal, which was an unexpected result. I am curious if I can tune the step sizes or number of iterations to remove this bimodal trend if I had more time. 

