%----------------------------------------------------------------------
% Conclusion

\begingroup
\allowdisplaybreaks

\section{Conclusion}

Inertial sensor calibration is an essential activity before installing an IMU into a self-driving vehicle for the purposes of navigation. Traditional calibration methods use simple tests on their respective rotational test beds and post-process the collected data in piece-meal sections to compute only the most basic of IMU error sources. In an effort to improve the status-quo, systematic calibration seeks to leverage batched estimation methods to extend model complexity and make use of all available information to estimate its model parameters. In addition, systematic calibration also establishes some measure of uncertainty for each model parameter which is useful for verifying the final calibration parameters used before vehicle application. 

A direct comparison between traditional and systematic calibration could not be achieved due to the rank-deficient model operators constructed for systematic calibration. This implies that the traditional motion profiles used on a three-axis rate table are not sufficient for systematic calibration, which brings into question what motion profiles are suitable for meaningful systematic calibration. 

In response, three motion profiles were created to test and evaluate of model accuracy and uncertainty are impacted. Overall, the second motion profile yielded the highest accuracy and smallest covariance, which was an unexpected result. This implies that more motion in a rotational test-bed is not always better. The key contributor to the second motion profile's performance is that the model parameters were subjected to smallest amount of correlation to each other. If someone were to continue the design of motion profiles to yield high-accuracy calibration, keeping model correlation in mind is key to success. 

Each of the three motion profiles were designed to produce full-rank model operators, but a fourth motion profile was created to purposely create another rank-deficient situation. It was shown that the model null space was non-trivial, and the model resolution matrix made clear which model parameters were affected. Two methods, truncated-SVD and Tikhonov regularization, were utilized to stabilize the solution. Each method yielded the same results which lead to the same error. Some model parameters were able to be resolved perfectly, but others were shown to be far from the truth. There may be situations were these dynamics are unavoidable, but this work shows that there are still methods to develop a partial solution. 

Overall, the benefits of systematic calibration over traditional calibration methods were demonstrated throughout this work. If careful consideration is given to the design of the motion profile, then systematic calibration can yield high-precision results with similar high-accuracy. In conclusion, this work provides a thorough proof-of-concept of strategies that could be adopted by inertial test laboratories to upgrade from traditional to systematic means of calibration.