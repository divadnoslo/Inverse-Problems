\documentclass[letterpaper]{article}

%----------------------------------------------------------------------
% Packages

\usepackage{amsmath}
\usepackage{amsfonts}
\usepackage{amssymb}
\usepackage{graphicx}
\usepackage{xcolor}
\usepackage{tikz, tcolorbox}
\usepackage{siunitx}
\usepackage{geometry}
\usepackage{fancyhdr}
\usepackage{lastpage}
\usepackage{placeins}


%----------------------------------------------------------------------
% Custom Math Commands

% General Shortcuts
\newcommand{\R}{\mathbb{R}} % Real Numbers
\newcommand{\C}{\mathbb{C}} % Complex Numbers
\newcommand{\N}{\mathbb{N}} % Natural Numbers (0, 1, 2, ...)
\newcommand{\Z}{\mathbb{Z}} % Integers (..., -2, -1, 0, 1, 2, ...)
\newcommand{\Q}{\mathbb{Q}} % Rational Numbers

% Linear Algebra Shortcuts
\newcommand{\bv}[1]{\boldsymbol{#1}}      % bold vector quantity
\newcommand{\T}{^{\intercal}}             % Transpose
\newcommand{\norm}[1]{|| {#1} ||_1}       % generic norm
\newcommand{\onenorm}[1]{|| {#1} ||_1}    % one norm
\newcommand{\twonorm}[1]{|| {#1} ||_2}    % two norm
\newcommand{\infnorm}[1]{|| {#1} ||_\inf}    % infinty norm

% Probability and Statistics Shortcuts
\newcommand{\E}[1]{\textrm{E}\left[#1\right]}
\newcommand{\Var}[1]{\textrm{Var}\left(#1\right)}
\newcommand{\Cov}[1]{\textrm{Cov}\left(#1\right)}

% Navigation Shortcuts
\newcommand{\navvec}[3]{\boldsymbol{#1}^{\,#2}_{#3}}              % navigation vector
\newcommand{\navmeas}[3]{\tilde{\boldsymbol{#1}}^{\,#2}_{#3}}     % navigation measurement
\newcommand{\navest}[3]{\hat{\boldsymbol{#1}}^{\,#2}_{#3}}        % navigation estimate
\newcommand{\navC}[2]{C^{\,#1}_{#2}}                              % coordinate frame transformation
\newcommand{\quat}[2]{q^{#1}_{#2}}                                % quaternion

% Misc.
\newcommand{\MATLAB}{MATLAB\,\textsuperscript{\tiny\textregistered}\,}


%----------------------------------------------------------------------
% Begin Document

\begin{document}
	
	
	%------------------------------------------------------------------
	% Title
	
	\title{Inverse Problems for Inertial Sensor Calibration}
	\author{David L. Olson}
	\date{\today}
	
	\maketitle
	
	%------------------------------------------------------------------
	% Abstract
	
	\begin{abstract}
	
	Inertial sensors are vital to navigation, guidance, and control (NGC) for a variety of vehicles. Inertial sensors such as accelerometers and gyroscopes measure specific and angular velocity respectively which are mechanized to produce a position, velocity, and attitude (PVA) solution \cite{groves2013principles}. All inertial sensors are subject to deterministic and stochastic error which require careful calibration and characterization before utilizing inertial sensors in a vehicle application. In industry, the current state of the art for inertial sensor calibration has made little advancement over the past couple decades, where calibration data is collected and post-processed using basic statistics to compute basic error model parameters. In the past, improved fidelity for calibration required purchasing higher and higher precision equipment, which eventually succumbs to the law of diminishing returns.
	
	However, ongoing research addressing inertial sensor calibration seeks to break dependence on high-cost high-precision test equipment by introducing advanced estimation techniques and stochastic filtering to maintain high fidelity calibration performance \cite{ImprovedIMUCalibrationProcedures,Rahimi763,8943630,9851744,aCalibrationMethodForSixAccelerometerINS}. While various papers only tackle small pieces of an entire calibration procedure, there is an opportunity to cast inertial sensor calibration as an inverse problem to provide a holistic methodology to the field. The forward problem is simply the process of transforming true dynamic inputs acting on the sensor to sensor outputs which is subject to error. Then, the inverse problem takes the form of calibration in which calibration parameters are estimated from collected sensor data. Using inverse problem techniques, calibration parameter estimation can advance beyond basic statistical techniques for higher fidelity performance to include covariance propagation to assign confidence to final calibration parameters. In addition, estimation may be performed in real-time during the collection of calibration data, alleviating logistical and programmatic troubles for an inertial test laboratory.
	
	The author plans to build upon previous work \cite{GeneralizedFramework} which homogenized multiple IEEE standards for varying accelerometer and gyroscope technologies \cite{IEEE_std_1293-2018,IEEE_std_1431-2004,IEEE_std_952-2020,IEEE_std_292-1969,IEEE_std_517-1974,IEEE_std_647-2006,IEEE_std_813-1988} which provide highly detailed inertial sensor error models. Inertial sensor data will be simulated using \cite{RTSim} which can provide true inertial sensor measurements which can then be corrupted with error and noise. Then, inverse problem techniques may be demonstrated and contrasted with traditional methods to showcase improvements and possible limitations to using these new highly detailed models. Higher fidelity calibration yields increased sensor accuracy, which ultimately enhances inertial navigation capabilities for a wide range of vehicle applications. 
	
\end{abstract}
	
	%------------------------------------------------------------------
	% Table of Contents
	
%	\tableofcontents
	
	%------------------------------------------------------------------
	% Header and Footer
	
	\thispagestyle{empty}
	\pagestyle{empty}
	
	\fancyhead{}
	\fancyfoot{}
	
	\thispagestyle{fancy}
	\pagestyle{fancy}
	
	\fancyhead[L]{\textbf{Inverse Problems}}
	\fancyhead[R]{\textbf{David L. Olson}}
	
	\fancyfoot[L]{\textbf{Project Idea}}
	\fancyfoot[R]{\thepage\ of \pageref{LastPage}}
	
	
	%------------------------------------------------------------------
	% Begin Technical Content
	
	
	
	%------------------------------------------------------------------
	% Print Bibliography
	
%	\newpage
%	\bibliographystyle{numeric}
%	\bibliography{references}
	
\end{document}


%----------------------------------------------------------------------
% Additional Copy-and-Paste Templates

% Figure

%\begin{figure}[h] \label{fig: my figure}
%	\centering
%	\includegraphics[width=0.8\textwidth]{./images/<image.png>}
%	\caption{\textcolor{red}{provide a caption}}
%\end{figure}


% Table

%\begin{center}
%	\begin{tabular} { | h1 | h2 | h3 | } \label{tab: my table}
	%		
	%		\hline
	%		x11 & x12 & x13 \\
	%		\hline
	%		x21 & x22 & x23 \\
	%		\hline
	%		x31 & x32 & x33 \\
	%		\hline
	%		
	%	\end{tabular}
%	\caption{\textcolor{red}{put caption here}
	%\end{center}
