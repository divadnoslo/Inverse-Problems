%----------------------------------------------------------------------
% Methods

\begingroup
\allowdisplaybreaks

\section{Methods}

 Consider an IMU under test strapped down to a three axis rotational test bed. The rotational test bed provides measurements of angular position and angular rate along each test axis. As exemplar hardware, the unit under test will be a STIM 300 IMU from Safran and the 210C Series Three Axis Position and Rate Table System from Ideal Aerosmith.


\subsection{STIM 300 IMU Specifications} \label{sec: stim 300 imu specifications}
 
The STIM 300 IMU, henceforth referred to as the unit under test (UUT), provides measurements of specific force and angular velocity. While the true forward model is unknown, it will be assumed that the basic forward model in equation \ref{eq: compact IMU forward error model} will sufficiently model the error of the device. Assuming the UUT uses the "10g" variant of accelerometers, bounds for the accelerometer-related model parameters are provided in table 5-5 of the specification sheet \cite{stim300SpecSheet}.
 
\begin{itemize}
	\item Bias: $|b_a| \leq 7.5 \times 10^{-3} g$
	\item Scale Factor Error : $|s_a| \leq 200 \,\textrm{ppm}$
	\item Misalignment: $|m_a| \leq 1\times 10^{-3} \,\textrm{rad}$ 
\end{itemize}
 
The accelerometers are also subject to zero-mean Gaussian noise with a velocity random walk (VRW) value of $0.07 \unit{\meter\per\second\per\sqrt\hertz}$, which translates to a $\sigma_a = \frac{0.07}{60} = 0.0012 \unit{\meter\per\second}$. Likewise, \cite{stim300SpecSheet} provides bounds for the gyroscopes in table 5-3. 
 
\begin{itemize}
	\item Bias: $|b_g| \leq 250 \unit{\degree\per\hour}$
	\item Scale Factor Error : $|s_g| \leq 500 \,\textrm{ppm}$
	\item Misalignment: $|m_g| \leq 1\times 10^{-3} \,\textrm{rad}$ 
\end{itemize}
 
The gyroscopes are also subject to zero-mean Gaussian noise with an angle random walk (ARW) value of $0.15 \unit{\degree\per\sqrt\hour}$ which translates to a $\sigma_g = \frac{0.15}{60}\frac{\pi}{180} = 4.3633 \times 10^{-5} \unit{\radian\per\second}$.

For simulation purposes, true model parameters will be selected within these bounds.


%\subsection{2013C Series Position and Rate Table System Specifications}
%
%The 2013C Series Position and Rate Table System, henceforth referred to as the test bed, is able to spin and point the UUT in all directions within Cartesian space. Per the specification sheet \cite{threeAxisRateTableTable}, each axis can spin with an accuracy of $\pm 0.001\%$ and point with an accuracy $\pm 15 \unit{\arcsecond}$.
%
%For simplicity, let measurements of angular position and angular rate be subject to zero-mean Gaussian noise with the inaccuracies above interpreted as the 3-sigma bound. For additionally simplicity, assume that each Euler angle representing the attitude of the UUT is subject to the root mean square of the test bed's pointing accuracy.
%
%\begin{align*}
%	\sigma_\theta \approx \sqrt{\frac{1}{3} 15 \unit{\arcsecond} + \frac{1}{3} 15 \unit{\arcsecond} + \frac{1}{3} 15 \unit{\arcsecond}} = 8.5 \times 10^{-3} \,\unit{\radian}
%\end{align*}
%
%Likewise, assume that the resulting angular velocity measurements of the UUT derived from the test bed are the root mean square of the test bed's spinning accuracy when spinning at $100 \unit{\degree\per\sec}$.
%
%\begin{align*}
%	\sigma_\omega \approx \frac{\pi}{180} \sqrt{\frac{1}{3} 0.1 + \frac{1}{3} 0.1 + \frac{1}{3} 0.1} = 9.6 \times 10^{-3} \,\unit{\radian\per\second}
%\end{align*}
%
%The approximations are very crude, however expressing their uncertainty as a Normal distribution allows for a consistent application in the methods ahead.


\subsection{Simulating Calibration Data and Model Parameters}

Rotational motion on the test bed will be simulated by developing a true sequence of angular velocities $\bv{\omega}$ across time $t$ experienced by the UUT. The attitude of the UUT, represented by the direction cosine matrix $C\left(t\right)$, will be integrated in discrete steps such that the $k^{\textrm{th}}$ step in the sequence is 

\begin{align}
	C\left(t_k\right) = C_0 \prod_{i = 1}^k e^{\left[\bv{\omega}\left(t_i\right) \times\right] \Delta t}
\end{align}  

where $\left[\bv{\omega}\left(t_i\right) \times\right]$ is a skew-symmetric matrix and $\Delta t$ is the time step. While the test bed rotates, the accelerometers will each be measuring components of the normal force from the test bed resulting from the acceleration due to gravity $g$. For each time step $k$, the true specific force quantity is given below.

\begin{align}
	\bv{f}\left(t_k\right) = C\left(t_k\right) \begin{bmatrix} 0 \\ 0 \\ g \end{bmatrix}
\end{align}

In the simulation, it will be assumed that quantities $\bv{\omega}$ and $\bv{f}$ will be available through measurement outputs of the test bed itself. Measurements of these quantities from the UUT will be corrupted with measurement error according to equations \ref{eq: expanded IMU forward error model}. The values of the model parameters are recorded in table \ref{tab: UUT simulated model parameters} and are selected with reasonable orders of magnitude as discussed in section \ref{sec: stim 300 imu specifications}. 

\begin{table}[h!]
	\centering
	\begin{tabular}{|p{4cm}|p{3.5cm}|p{3.5cm}|}
		\hline
		\textbf{Model Parameter} & \textbf{Accel Values} & \textbf{Gyro Values} \\ \hline
		Fixed Bias X & $0.0628 \,\unit{\meter\per\second\squared}$ & $0.48481 \times 10^{-3} \,\unit{\radian\per\second}$ \\ \hline
		Fixed Bias Y & $-0.0510 \,\unit{\meter\per\second\squared}$ & $0.14544 \times 10^{-3}  \,\unit{\radian\per\second}$ \\ \hline
		Fixed Bias Z & $0.0363 \,\unit{\meter\per\second\squared}$ & $1.2120 \times 10^{-3}  \,\unit{\radian\per\second}$ \\ \hline
		Scale Factor Error X & $150 \,\textrm{ppm}$ & $450 \,\textrm{ppm}$ \\ \hline
		Scale Factor Error Y & $-175 \,\textrm{ppm}$ & $-300 \,\textrm{ppm}$ \\ \hline
		Scale Factor Error Z & $198 \,\textrm{ppm}$ & $175 \,\textrm{ppm}$ \\ \hline
		Misalignment XY & $0.1 \,\unit{\milli\radian}$ & $-0.1 \,\unit{\milli\radian}$ \\ \hline
		Misalignment XZ & $-0.2 \,\unit{\milli\radian}$ & $0.2 \,\unit{\milli\radian}$ \\ \hline
		Misalignment YX & $0.3 \,\unit{\milli\radian}$ & $-0.3 \,\unit{\milli\radian}$ \\ \hline
		Misalignment YZ & $-0.4 \,\unit{\milli\radian}$ & $0.4 \,\unit{\milli\radian}$ \\ \hline
		Misalignment ZX & $0.5 \,\unit{\milli\radian}$ & $-0.5 \,\unit{\milli\radian}$ \\ \hline
		Misalignment ZY & $-0.6 \,\unit{\milli\radian}$ & $0.6 \,\unit{\milli\radian}$ \\ \hline
	\end{tabular}
	\caption{UUT Simulated Model Parameters}
	\label{tab: UUT simulated model parameters}
\end{table}
\FloatBarrier

Capabilities for simulating calibration sequences on test beds and IMU error models have been developed in \MATLAB. 


\subsection{Formulating IMU Calibration as a Discrete Linear Inverse Problem}

Recall the system of equations from equation \ref{eq: expanded IMU forward error model} and consider the $x$-axis accelerometer measurements.

\begin{align*}
	\tilde{f}_x = \left(1 + s_{a,x}\right) f_x + m_{a,xy} f_y + m_{a,xz} f_z + b_{a,x}
\end{align*}

The equation above can be re-arranged to express the model parameters as a function of the accelerometer error $\Delta f = \tilde{f} - f$. 

\begin{align*}
	\tilde{f}_x &= \left(1 + s_{a,x}\right) f_x + m_{a,xy} f_y + m_{a,xz} f_z + b_{a,x} \\
	\\
	\tilde{f}_x - f_x &= b_{a,x} + s_{a,x} f_x + m_{a,xy} f_y + m_{a,xz} f_z \\
	\\
	\Delta f_x &= b_{a,x} + s_{a,x} f_x + m_{a,xy} f_y + m_{a,xz} f_z
\end{align*}

Assuming that both the UUT and test bed are able to provide synchronized measurements at the same sampling frequency, a series of measurements from these devices can be organized into another system of equations. 

\begin{align} \label{eq: expanded block of GM = d}
	\begin{bmatrix} 
		1 & f_x[1] & f_y[1] & f_z[1] \\ 1 & f_x[2] & f_y[2] & f_z[2] \\ \vdots & \vdots & \vdots & \vdots \\ 1 & f_x[m] & f_y[m] & f_z[m]
	\end{bmatrix} \begin{bmatrix}
		b_{a,x} \\ s_{a,x} \\ m_{a,xy} \\ m_{a,xz}
	\end{bmatrix} = \begin{bmatrix}
		\Delta f_x[1] \\ \Delta f_x[2] \\ \vdots \\ \Delta f_x[m]
	\end{bmatrix}
\end{align}

This system of equations can be expressed in the form $G\bv{m} = \bv{d}$. In this expression, "true" measurements $f_x[n],\,f_y[n],\,f_z[n]$ within the model operator are computed from measurements provided by the test bed, and the model parameters are various calibration factors from the forward model. Elements of the data vector $\bv{d}$ are the difference of the UUT output and test bed output such that $d[n] = \tilde{f}_x[n] - f_x[n]$.

Let $F$ be the model operator demonstrated by equation \ref{eq: expanded block of GM = d}.

\begin{align} \label{eq: F}
	F &\defeq \begin{bmatrix} 
		1 & f_x[1] & f_y[1] & f_z[1] \\ 1 & f_x[2] & f_y[2] & f_z[2] \\ \vdots & \vdots & \vdots & \vdots \\ 1 & f_x[m] & f_y[m] & f_z[m]
	\end{bmatrix},\,\,\, F \in \R^{m \times 4}
\end{align}

Then, the discrete linear inverse problem for all accelerometer calibration parameters given in equation \ref{eq: expanded IMU forward error model} can defined below.

\begin{align} \label{eq: Gm = d for all accel parameters}
	G_a \bv{m}_a &= \bv{d}_a \notag\\
	\\
	\begin{bmatrix} 
		F & 0_{m \times 4} & 0_{m \times 4} \\
		0_{m \times 4} & F & 0_{m \times 4} \\
		0_{m \times 4} & 0_{m \times 4} & F \\
	\end{bmatrix} \begin{bmatrix}
		b_{a,x} \\ s_{a,x} \\ m_{a,xy} \\ m_{a,xz} \\ b_{a,y} \\ m_{a,yz} \\ s_{a,y} \\ m_{a,yz} \\ b_{a,z} \\ m_{a,zx} \\ m_{a,zy} \\ s_{a,z}
	\end{bmatrix} &= \begin{bmatrix}
		\Delta f_x[1] \\ \vdots \\ \Delta f_x[m] \\ \\ \Delta f_y[1] \\ \vdots \\ \Delta f_y[m] \\ \\ \Delta f_z[1] \\ \vdots \\ \Delta f_z[m]
	\end{bmatrix} \notag
\end{align}

Likewise, let $\Omega$ be the model operator for all gyroscope measurements specific to one sensor. 

\begin{align} \label{eq: Omega}
	\Omega &\defeq \begin{bmatrix} 
		1 & \omega_x[1] & \omega_y[1] & \omega_z[1] \\ 1 & \omega_x[2] & \omega_y[2] & \omega_z[2] \\ \vdots & \vdots & \vdots & \vdots \\ 1 & \omega_x[m] & \omega_y[m] & \omega_z[m]
	\end{bmatrix},\,\,\, \Omega \in \R^{m \times 4}
\end{align}

Then, the discrete linear inverse problem for all gyroscope calibration parameters given in equation \ref{eq: expanded IMU forward error model} can defined below.

\begin{align} \label{eq: Gm = d for all gyro parameters}
	G_g \bv{m}_g &= \bv{d}_g \notag\\
	\\
	\begin{bmatrix} 
		\Omega & 0_{m \times 4} & 0_{m \times 4} \\
		0_{m \times 4} & \Omega & 0_{m \times 4} \\
		0_{m \times 4} & 0_{m \times 4} & \Omega \\
	\end{bmatrix} \begin{bmatrix}
		b_{g,x} \\ s_{g,x} \\ m_{g,xy} \\ m_{g,xz} \\ b_{g,y} \\ m_{g,yz} \\ s_{g,y} \\ m_{g,yz} \\ b_{g,z} \\ m_{g,zx} \\ m_{g,zy} \\ s_{g,z}
	\end{bmatrix} &= \begin{bmatrix}
		\Delta \omega_x[1] \\ \vdots \\ \Delta \omega_x[m] \\ \\ \Delta \omega_y[1] \\ \vdots \\ \Delta \omega_y[m] \\ \\ \Delta \omega_z[1] \\ \vdots \\ \Delta \omega_z[m]
	\end{bmatrix} \notag
\end{align}


\subsection{Assessing Singular Values}

Prior to solving the inverse problem, we must ensure that the problem is not ill-conditioned. This is accomplished by performing a singular value decomposition (SVD) on the model operators $\Omega$ and $F$ respectively from equations \ref{eq: Omega} and \ref{eq: F}. Although these model operators may appear to be full-rank when calling \MATLAB's \verb|rank()| function, some singular values may be poorly scaled. Therefore, it is essential to verify that the singular values are suitable for each new motion profile run on the three-axis rate table. 

If the singular values are poorly scaled, it may be necessary to compute a truncated SVD solution or regularize the solution. In both cases, this takes away the opportunity to assess the resulting covariances and correlations between model parameters as the model null-space is now non-trivial and may lead to a biases solution in the model parameters. 


\subsection{L2 Regression without a Known Normal Distribution} \label{sec: L2 regresssion}

Although a typical IMU specification sheet contains values of VRW and ARW which relate to the white noise present on each sensor channel, in practice each individual sensor is subject to its own amount of white noise. Therefore, to be consistent with application, we will assume that each sample is dependent and identically distributed, although the standard deviation associated with the white noise will be unknown. Therefore, casting this problem as a weighted least squares problem is not feasible. Additionally in practice, it is not a safe assumption to assume that the power of the white noise signal is equally distributed at all frequencies, introducing noise color, however this will be ignored for both simulation and evaluation purposes. 

Since the inverse problem can not be casted as a weighted least squares problem, the inverse problem will be solved via the normal equations (assuming the model operator is not ill-conditioned) without any scaling applied to the model operators. 

\begin{align} \label{eq: L2 Regression}
	\bv{m}_{a,L2} &= \left(G_a^T G_a\right)^{-1} G_a^T \bv{d_a} \notag\\
	\\
	\bv{m}_{g,L2} &= \left(G_g^T G_g\right)^{-1} G_g^T \bv{d_g} \notag
\end{align}

Without a known standard deviation, we instead must estimate the resulting standard deviation from the residuals after performing the least squares fit. 

\begin{align} \label{eq: estimated sigma}
	s_a = \frac{\twonorm{G_a \bv{m}_{a,L2} - \bv{d}_a}}{\sqrt{m - n}} \notag\\
	\\
	s_g = \frac{\twonorm{G_g \bv{m}_{g,L2} - \bv{d}_g}}{\sqrt{m - n}} \notag
\end{align}

With these estimated standard deviations, we can approximate the model covariance. 

\begin{align} \label{eq: estimated model covaraince}
	\tilde{C}_a = s_a^2 \left(G_a^T G_a\right)^{-1} \notag\\
	\\
	\tilde{C}_g = s_g^2 \left(G_g^T G_g\right)^{-1} \notag
\end{align}

Rather than using a Normal distribution to determine the 95\% confidence interval, we must use the Student's $t$ distribution instead. In this use case, the degree of freedom with be very large which should well-approximate the normal distribution. 

\begin{align} \label{eq: confidence intervals}
	m_{a,L2_i} &\pm t_{0.95,\nu} \sqrt{\tilde{C}_{a_{i,i}}}  \notag\\
	\\
	m_{g,L2_i} &\pm t_{0.95,\nu} \sqrt{\tilde{C}_{g_{i,i}}}  \notag
\end{align}

In addition to the model covariance, the correlation between each model parameter will also be examined. 


\subsection{Motion Profile Evaluation}

First, we will investigate using traditional calibration data to solve the inverse problem and make comparisons to the traditional post-processing methods for IMU calibration. 

Then, we will investigate three unique motion profiles for evaluation.

\begin{enumerate}
	\item \textbf{Single-Axis Tilts in One Direction}
	\item \textbf{Single-Axis Tilts in Two Directions}
	\item \textbf{Multi-Axis Tilts in All Directions}
\end{enumerate}
	

The dynamics of each motion profile will cause varying model fits, covariances, and correlations. These insights will be useful to those needing to develop new motion profiles to calibrate IMUs. Their Euler angle profiles are proved in figures \ref{fig: MP1 Euler Angle Profile}, \ref{fig: MP2 Euler Angle Profile}, and \ref{fig: MP3 Euler Angle Profile}.

\begin{figure}[h] 
	\centering
	\includegraphics[width=0.5\textwidth]{./images/MP1_euler_angle_profile.eps}
	\caption{Motion Profile 1: Euler Angle Profile}
	\label{fig: MP1 Euler Angle Profile}
\end{figure}
\FloatBarrier

\begin{figure}[h] 
	\centering
	\includegraphics[width=0.5\textwidth]{./images/MP2_euler_angle_profile.eps}
	\caption{Motion Profile 2: Euler Angle Profile}
	\label{fig: MP2 Euler Angle Profile}
\end{figure}
\FloatBarrier

\begin{figure}[h] 
	\centering
	\includegraphics[width=0.5\textwidth]{./images/MP3_euler_angle_profile.eps}
	\caption{Motion Profile 3: Euler Angle Profile}
	\label{fig: MP3 Euler Angle Profile}
\end{figure}
\FloatBarrier

Then, we will analyze a motion profile designed to purposely be ill-conditioned to determine what model fit, if any, could be obtained if forced into this situation. This motion profile is provided in figure \ref{fig: MP4 Euler Angle Profile}.

\begin{figure}[h] 
	\centering
	\includegraphics[width=0.5\textwidth]{./images/MP4_euler_angle_profile.eps}
	\caption{Motion Profile 3: Euler Angle Profile}
	\label{fig: MP4 Euler Angle Profile}
\end{figure}
\FloatBarrier

