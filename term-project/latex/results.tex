%----------------------------------------------------------------------
% Results

\begingroup
\allowdisplaybreaks

\section{Results}

The following sections contain the results of each motion profile evaluation.


\subsection{Traditional Calibration Sequence}

A traditional calibration sequence was simulated for a UUT with an assumed sample rate of $100 \,\unit{\hertz}$. Each test specified in table \ref{tab: traditional_calibration_tests} for a duration of $10 \,\unit{\second}$. All of the accelerometer test data were appended together to build the model operator $G_a$, and as well as all of the gyroscope test data to build $G_g$. Each model operator contains 12 columns. Before solving the least squares problem, the singular values were assessed for each model operator as shown in figures \ref{fig: MP0 gyro singular values} and \ref{fig: MP0 accel singular values}.

\begin{figure}[h] 
	\centering
	\includegraphics[width=0.5\textwidth]{./images/MP0_gyro_singular_values.eps}
	\caption{Traditional Motion Profile - Gyroscope Singular Values}
	\label{fig: MP0 gyro singular values}
\end{figure}
\FloatBarrier

\begin{figure}[h] 
	\centering
	\includegraphics[width=0.5\textwidth]{./images/MP0_accel_singular_values.eps}
	\caption{Traditional Motion Profile - Accelerometer Singular Values}
	\label{fig: MP0 accel singular values}
\end{figure}
\FloatBarrier

For both $G_g$ and $G_a$, there are six singular values that equal $0$, meaning that each model operator is not full rank! This means the products $G_g^T G_g$ and $G_a^T G_a$ are not invertable, meaning that the normal equations can not be solved. Rather than trying to regularize the solution, it was decided to instead abandon this comparison and instead evaluate new motion profiles for the three-axis rate table that will result in full rank model operators. 


\subsection{New Motion Profile Evaluation}

In light of the fact that the traditional motion profile does not yield a full rank model operator, new motion profiles were investigated. It would take a significantly long time to exhaust every possible motion profile for rotating an IMU through every possible attitude, therefore three motion profiles were selected for evaluation. 

\begin{enumerate}
	\item \textbf{Single-Axis Tilts in One Direction}
	\item \textbf{Single-Axis Tilts in Two Directions}
	\item \textbf{Multi-Axis Tilts in All Directions}
\end{enumerate}

The following subsections contain results for each motion profile. 


\subsubsection{Motion Profile 1: Single-Axis Tilts in One Direction}

This motion profile resulted in the following angular velocity profile and gyroscope measurement error shown in the figure below. 

\begin{figure}[!h] 
	\centering
	\includegraphics[width=0.49\textwidth]{./images/MP1_angular_velocity_profile.eps} \hfill
	\includegraphics[width=0.49\textwidth]{./images/MP1_angular_velocity_error.eps}
	\caption{Motion Profile 1 - Angular Velocity Profile}
	\label{fig: MP1 angular velocity profile}
\end{figure}
\FloatBarrier

In addition, the resulting specific force profile due to the rotation of the IMU through the local gravity field is shown in the figure below. 

\begin{figure}[!h] 
	\centering
	\includegraphics[width=0.49\textwidth]{./images/MP1_specific_force_profile.eps} \hfill
	\includegraphics[width=0.49\textwidth]{./images/MP1_specific_force_error.eps}
	\caption{Motion Profile 1 - Specific Force Profile}
	\label{fig: MP1 specific force profile}
\end{figure}
\FloatBarrier

Prior to solving the normal equations, the singular values for both model operators $G_g$ and $G_a$ were visually inspected to ensure each were well-conditioned. 

\begin{figure}[!h] 
	\centering
	\includegraphics[width=0.49\textwidth]{./images/MP1_gyro_singular_values.eps} \hfill
	\includegraphics[width=0.49\textwidth]{./images/MP1_accel_singular_values.eps}
	\caption{Motion Profile 1 - Singular Values}
	\label{fig: MP1 singular values}
\end{figure}
\FloatBarrier

Following the process discussed in section \ref{sec: L2 regresssion}, the following results were achieved for the gyroscope and accelerometer calibrations. 

\begin{figure}[h] 
	\centering
	\includegraphics[width=1\textwidth]{./images/MP1_gyro_results.png}
	\caption{Motion Profile 1 - Gyroscope Calibration Results}
	\label{fig: MP1 gyro results}
\end{figure}
\FloatBarrier

\begin{figure}[h] 
	\centering
	\includegraphics[width=1\textwidth]{./images/MP1_accel_results.png}
	\caption{Motion Profile 1 - Accelerometer Calibration Results}
	\label{fig: MP1 accel results}
\end{figure}
\FloatBarrier

In addition, the following model correlation matrices were achieved.

\begin{figure}[!h] 
	\centering
	\includegraphics[width=0.49\textwidth]{./images/MP1_gyro_correlation_matrix.eps} \hfill
	\includegraphics[width=0.49\textwidth]{./images/MP1_accel_correlation_matrix.eps}
	\caption{Motion Profile 1 - Correlation Matrices}
	\label{fig: MP1 correlation matrix}
\end{figure}
\FloatBarrier

\subsubsection{Motion Profile 2: Single-Axis Tilts in Both Directions}

This motion profile resulted in the following angular velocity profile and gyroscope measurement error shown in the figure below. 

\begin{figure}[!h] 
	\centering
	\includegraphics[width=0.49\textwidth]{./images/MP2_angular_velocity_profile.eps} \hfill
	\includegraphics[width=0.49\textwidth]{./images/MP2_angular_velocity_error.eps}
	\caption{Motion Profile 2 - Angular Velocity Profile}
	\label{fig: MP2 angular velocity profile}
\end{figure}
\FloatBarrier

In addition, the resulting specific force profile due to the rotation of the IMU through the local gravity field is shown in the figure below. 

\begin{figure}[!h] 
	\centering
	\includegraphics[width=0.49\textwidth]{./images/MP2_specific_force_profile.eps} \hfill
	\includegraphics[width=0.49\textwidth]{./images/MP2_specific_force_error.eps}
	\caption{Motion Profile 2 - Specific Force Profile}
	\label{fig: MP2 specific force profile}
\end{figure}
\FloatBarrier

Prior to solving the normal equations, the singular values for both model operators $G_g$ and $G_a$ were visually inspected to ensure each were well-conditioned. 

\begin{figure}[!h] 
	\centering
	\includegraphics[width=0.49\textwidth]{./images/MP2_gyro_singular_values.eps} \hfill
	\includegraphics[width=0.49\textwidth]{./images/MP2_accel_singular_values.eps}
	\caption{Motion Profile 2 - Singular Values}
	\label{fig: MP2 singular values}
\end{figure}
\FloatBarrier

Following the process discussed in section \ref{sec: L2 regresssion}, the following results were achieved for the gyroscope and accelerometer calibrations. 

\begin{figure}[h] 
	\centering
	\includegraphics[width=1\textwidth]{./images/MP2_gyro_results.png}
	\caption{Motion Profile 2 - Gyroscope Calibration Results}
	\label{fig: MP2 gyro results}
\end{figure}
\FloatBarrier

\begin{figure}[h] 
	\centering
	\includegraphics[width=1\textwidth]{./images/MP2_accel_results.png}
	\caption{Motion Profile 2 - Accelerometer Calibration Results}
	\label{fig: MP2 accel results}
\end{figure}
\FloatBarrier

In addition, the following model correlation matrices were achieved.

\begin{figure}[!h] 
	\centering
	\includegraphics[width=0.49\textwidth]{./images/MP2_gyro_correlation_matrix.eps} \hfill
	\includegraphics[width=0.49\textwidth]{./images/MP2_accel_correlation_matrix.eps}
	\caption{Motion Profile 2 - Correlation Matrices}
	\label{fig: MP2 correlation matrix}
\end{figure}
\FloatBarrier

\subsubsection{Motion Profile 3: Multi-Axis Tilts in All Directions}

This motion profile resulted in the following angular velocity profile and gyroscope measurement error shown in the figure below. 

\begin{figure}[!h] 
	\centering
	\includegraphics[width=0.49\textwidth]{./images/MP3_angular_velocity_profile.eps} \hfill
	\includegraphics[width=0.49\textwidth]{./images/MP3_angular_velocity_error.eps}
	\caption{Motion Profile 3 - Angular Velocity Profile}
	\label{fig: MP3 angular velocity profile}
\end{figure}
\FloatBarrier

In addition, the resulting specific force profile due to the rotation of the IMU through the local gravity field is shown in the figure below. 

\begin{figure}[!h] 
	\centering
	\includegraphics[width=0.49\textwidth]{./images/MP3_specific_force_profile.eps} \hfill
	\includegraphics[width=0.49\textwidth]{./images/MP3_specific_force_error.eps}
	\caption{Motion Profile 3 - Specific Force Profile}
	\label{fig: MP3 specific force profile}
\end{figure}
\FloatBarrier

Prior to solving the normal equations, the singular values for both model operators $G_g$ and $G_a$ were visually inspected to ensure each were well-conditioned. 

\begin{figure}[!h] 
	\centering
	\includegraphics[width=0.49\textwidth]{./images/MP3_gyro_singular_values.eps} \hfill
	\includegraphics[width=0.49\textwidth]{./images/MP3_accel_singular_values.eps}
	\caption{Motion Profile 3 - Singular Values}
	\label{fig: MP3 singular values}
\end{figure}
\FloatBarrier

Following the process discussed in section \ref{sec: L2 regresssion}, the following results were achieved for the gyroscope and accelerometer calibrations. 

\begin{figure}[h] 
	\centering
	\includegraphics[width=1\textwidth]{./images/MP3_gyro_results.png}
	\caption{Motion Profile 3 - Gyroscope Calibration Results}
	\label{fig: MP3 gyro results}
\end{figure}
\FloatBarrier

\begin{figure}[h] 
	\centering
	\includegraphics[width=1\textwidth]{./images/MP3_accel_results.png}
	\caption{Motion Profile 3 - Accelerometer Calibration Results}
	\label{fig: MP3 accel results}
\end{figure}
\FloatBarrier

In addition, the following model correlation matrices were achieved.

\begin{figure}[!h] 
	\centering
	\includegraphics[width=0.49\textwidth]{./images/MP3_gyro_correlation_matrix.eps} \hfill
	\includegraphics[width=0.49\textwidth]{./images/MP3_accel_correlation_matrix.eps}
	\caption{Motion Profile 3 - Correlation Matrices}
	\label{fig: MP3 correlation matrix}
\end{figure}
\FloatBarrier


\subsection{IMU Calibration for Ill-Conditioned System Dynamics}

A fourth motion profile was crafted to intentionally create an ill-conditioned model operator that could not used in the methods covered in section \ref{sec: L2 regresssion}. This motion profile resulted in the following angular velocity profile and gyroscope measurement error shown in the figure below. 

\begin{figure}[!h] 
	\centering
	\includegraphics[width=0.49\textwidth]{./images/MP4_angular_velocity_profile.eps} \hfill
	\includegraphics[width=0.49\textwidth]{./images/MP4_angular_velocity_error.eps}
	\caption{Motion Profile 4 - Angular Velocity Profile}
	\label{fig: MP4 angular velocity profile}
\end{figure}
\FloatBarrier

In this case, the ill-conditioned model operator only applied to the gyroscope calibration, therefore we will only cover the results from this sensor suite. 

\begin{figure}[h] 
	\centering
	\includegraphics[width=0.5\textwidth]{./images/MP4_gyro_singular_values.eps}
	\caption{Motion Profile 4 - Gyroscope Singular Values}
	\label{fig: MP4 gyro singular values}
\end{figure}
\FloatBarrier

This is the situation when both the data and model null spaces are non-trivial, as the model operator contains 12 columns but is only of rank 9. The model null space is shown below in figure \ref{fig: MP4 model null space}. 

\begin{figure}[h] 
	\centering
	\includegraphics[width=0.35\textwidth]{./images/model_null_space.png}
	\caption{Motion Profile 4 - Model Null Space}
	\label{fig: MP4 model null space}
\end{figure}
\FloatBarrier

The impact of this null space is also shown in the model resolution matrix, which is provided in figure \ref{fig: MP4 gyro model resolution}. Notice that every odd index (i.e., 1, 3, 5, ...) of the model parameters are perfect resolved, but each even index (i.e., 2, 4, 6, ...) are subject to some sort of smearing. This is likely due to the specific dynamics chosen for this motion profile.

\begin{figure}[h] 
	\centering
	\includegraphics[width=0.5\textwidth]{./images/MP4_gyro_model_resolution.eps}
	\caption{Motion Profile 4 - Gyroscope Model Resolution}
	\label{fig: MP4 gyro model resolution}
\end{figure}
\FloatBarrier

The generalized pseudo-inverse is used to solve for a model of calibration parameters. 

\begin{align}
	\bv{m}_{SVD} = G_g^\dagger \bv{d}_g = V_p S_p^{-1} U_p^T \bv{d}_g
\end{align}

Another means to solve for a model of calibration parameters is to apply Tikhonov regularization. In this case $0^{th}$-order Tikhonov regularization, also known as damped least squares, was utilized. To select the damping parameter $\alpha$, an L-curve was created to find the point at which both the model residuals and model norm are both minimized. 

\begin{figure}[h] 
	\centering
	\includegraphics[width=0.5\textwidth]{./images/MP4_accel_0th_order_tik_l_curve.eps}
	\caption{Motion Profile 4 - Tikhonov Regularization L-Curve}
	\label{fig: MP4 gyro tikh l-curve}
\end{figure}
\FloatBarrier

This point results in a value of $\alpha = 0.007824$. The solution $\bv{m}_{tikh}$ is computed accordingly below. 

\begin{align}
	\bv{m}_{tikh} = \left(G_g^T G_g + \alpha I\right)^{-1} G_g^T \bv{d}_g
\end{align}

The results for the two methods of solving for a calibration model in the presence of ill-conditioned model operators are below. 

\begin{figure}[h] 
	\centering
	\includegraphics[width=1\textwidth]{./images/MP4_gyro_results.png}
	\caption{Motion Profile 4 - Gyroscope Calibration Results}
	\label{fig: MP4 gyro results}
\end{figure}
\FloatBarrier

Take notice that each method produced the same model. 

