\begin{abstract}
	
	Inertial sensors are vital to navigation, guidance, and control (NGC) for a variety of vehicles. Inertial sensors such as accelerometers and gyroscopes measure specific and angular velocity respectively which are mechanized to produce a position, velocity, and attitude (PVA) solution \cite{groves2013principles}. All inertial sensors are subject to multiple error sources which requires careful calibration to compensate raw sensor measurements prior to use. Today, inertial sensors are generally strapped down to laboratory test equipment such as a three-axis rate table and are subjected to precise pre-planned motion sequences. Collected data during these tests are post-processed, often chopped into piecemeal segments, and evaluated in a vacuum using basic statistics. 
	
	Ongoing academic research for inertial sensor calibration seeks to break dependence from utilizing high-cost high-precision test equipment by using more sophisticated estimation techniques and stochastic filtering. 
	
	There is substantial room for improvement to inertial sensor calibration by formulating the approach as an inverse problem. Given collected calibration data, model parameters associated with inertial sensor error may be estimated with respect to pre-determined formulas. Opportunities for improvement which are not standard in industry include assigning confidence intervals to model parameters as well as sensitivity to sensor noise. There is also room to determine which model parameters are observable for a given motion sequence, which can inform the limitations of standard calibration procedures and lead to the development of improved procedures.
	
	The author plans to build upon previous work which homogenized IEEE standards for varying accelerometer and gyroscope technologies which will provide well understood formulas and model parameter definitions. \cite{GeneralizedFramework} Inertial sensor data will be simulated using \cite{RTSim} which can provide true inertial sensor measurements which can then be corrupted with error and noise. 
	
\end{abstract}