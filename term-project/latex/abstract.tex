\begin{abstract}
	
	Inertial sensors are vital to navigation, guidance, and control for a variety of vehicles such as self-driving cars. Inertial sensors, i.e.,  accelerometers and gyroscopes, measure specific and angular velocity respectively. These measurements are propagated to produce a position, velocity, and attitude solution for the vehicle, all three of which are necessary for proper vehicle navigation. Like any other sensor, all inertial sensors are subject to both deterministic and stochastic error which require careful calibration and characterization before use in any vehicle. In industry, the current state of the art for inertial sensor calibration has made little advancement over the past couple decades. Traditionally, hours of high-rate inertial sensor data is collected in static poses or constant angular rates and post-processed in piecemeal sections which average over these long periods of collected data to compute basic error model parameters. These traditional methods provide no ability to assess any uncertainty in the computed parameters, making any judgments about the fidelity of the calibration subjective. Regardless, instead of investigating advanced algorithms for estimating calibration parameters, industry achieved for high-fidelity calibrations solely via purchasing higher and higher precision equipment, which eventually succumbs to the law of diminishing returns.
	
	However, ongoing academic research addressing inertial sensor calibration seeks to break dependence on high-cost high-precision test equipment by introducing advanced estimation techniques, often referred to as systematic calibration. While various papers only tackle small pieces of an entire calibration procedure, there is an opportunity to cast inertial sensor calibration as an inverse problem to provide a holistic methodology to the field. The forward problem is the transformation of true dynamic inputs acting on the sensor into sensor outputs which are subject to error. Then, the inverse problem takes the form of calibration in which the model parameters are estimated from collected sensor data. Using inverse problem techniques, calibration parameter estimation can advance beyond these piecemeal strategies to achieve a higher-fidelity solution as well as provide metrics to assess the uncertainty and correlation between model calibration parameters.
	
	This work focused on posing inertial sensor calibration as an inverse problem which included modeling inertial sensor error and rotational test bed dynamics. This provided an opportunity to employ linear inverse problem techniques, even in the presence of ill-conditioned model operators caused by the modeled table dynamics. In addition, the model covariance and correlation between parameters was also analyzed, which is rarely reported in any academic literature. In summary, this work highlighted both the benefits and additional considerations needed for systematic calibration.
		
\end{abstract}